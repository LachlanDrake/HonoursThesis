
\chapter{Conclusion and Extensions}

\section{Conclusion}
This thesis explored the main concepts necessary for the implementation of an autonomous multi-rotor unmanned aerial vehicle. First, an accurate mathematical model of the system dynamics was established and system parameters were estimated. This model was implemented in MATLAB Simulink, such that various control techniques could be simulated. A stable and accurate control system was progressively developed by iterative testing and improvement. Initially PID control was implemented, followed by a backstepping controller and finally an integral backstepping controller. Actuator saturation was then accounted for with use of a sigmoidal function. This control system was tested and found to adequately stabilise the system and also allow positional commands to be followed. Next, state estimation techniques were explored with the use of various sensors in an extended Kalman filter (EKF) algorithm. In particular two algorithms were developed and tested, with one using Global Positioning System (GPS) technology and the other using an optical flow sensor for use in GPS-limited environments. The control system was simulated using the EKF-estimated states and confirmed to be adequately stable. Finally, a UAV hardware platform was configured with open source flight control software (ArduPilot) and flown such that the EKF algorithm could be tested on real flight data gathered from the sensors.

\section{Future Extensions}
There are many possible extensions to this project, for example:

\begin{itemize}
\item Implement the backstepping control system on the hardware platform.
\item Account for motor/propeller failure events during flight.
\item Extend sensor fusion algorithm to handle sensor failure.
\item Implement obstacle avoidance.
\item Analyse the usefulness of additional sensors e.g differential pressure sensor (for airspeed).
\item Consider the impacts of other effects e.g air resistance and gyroscopic effects.
\item Consider slew rates of actuators, propeller inertia, etc.
\item Develop a quaternion-based model to avoid singularities.
\item Account for sensor biases.


\end{itemize}

\clearpage


