
\chapter{Conclusion and Extensions}

\section{Conclusion}
The aim of this thesis was to apply the concepts of modelling, control and state estimation to a specific configuration of rotary-wing aircraft to enable autonomous flight. In particular, Newton-Euler formalism was applied to the multi-rotor configuration to derive an accurate model for simulation. Both linear and nonlinear control techniques were developed and tested in simulation. State estimation algorithms based upon the extended Kalman filter were developed with a particular focus on examining the use of alternative sensors in GPS denied environments. The state estimation techniques were demonstrated in simulation and also tested on data gathered by sensors onboard a hexacopter platform during autonomous flights.\\


The initial model of the physical system was developed as a general multi-rotor model, i.e. this model could be applied to any number of configurations of multi-rotor vehicles. The total vertical thrust and each of the axial torques were taken as the four inputs to the system. However, the system has six degrees of freedom - three rotational dimensions and three translational dimensions - and hence is underactuated. This presents a challenging control problem as it is impossible to independently control horizontal motion in this case - rotational motion is first necessary to enact horizontal motion. To complete the model, the specific configuration of the hexacopter was explored. This allowed a more accurate model of the hardware platform to be developed and tested in simulation. Further, the parameters of the hardware platform were either measured or estimated based on external data such that the simulated model would adequately represent the dynamics of the physical system. The derivation of the model did not include any approximations or linearisations, however it did not necessarily account for all possible effects e.g wind, gyroscopic effects, air resistance. Previous studies demonstrate that these effects can be neglected without significant impacts to the accuracy of the model. Therefore, these effects were left out of the model to ensure the model was not unnecessarily complicated.\\

In the pursuit of autonomous flight, control theory is one of the most significant areas of research. A robust control system with a large operating region is vital to ensuring the success of an autonomous aircraft. The control problem was originally approached using PID control laws and these proved to be useful for hovering flight with minimal disturbances. However, since this control system relies on linearisation of the model about an operating point, the limitations become apparent if the vehicle is disturbed significantly from the assumed operating region. To improve upon this, backstepping - a nonlinear control technique - was employed in simulation. This resulted in improved operation, but nonzero steady state error was observed. The next progression was a set of backstepping control laws which also utilised integral action in the position controllers. This eliminated the previously observed steady state error whilst also increasing the robustness of the controller and limiting the effects of model uncertainties. The final problem was that of actuator saturation resulting in loss of control. This was countered by limiting the output of the position controllers with the use of a sigmoid function. The resulting controller was demonstrably robust in simulation in the presence of external disturbances. In summary, the resulting control system was able to respond to large positional commands and move to the given coordinates even in the presence of disturbances and model uncertainties.\\

An equally challenging problem for autonomous vehicles is that of state estimation. For an aircraft to accurately move to a given coordinate it must have a system to interpret sensor data to estimate its translational and rotaional states. Most modern systems rely on GPS data to estimate position - however GPS data is not always available and therefore an increasing amount of research is dedicated to state estimation in GPS denied environments. For this purpose two algorithms were proposed, with one utilising a GPS sensor and one instead making use of an optical flow sensor. Both algorithms were based upon the extended Kalman filter and used a number of common sensors including a barometer, IMU and compass. The algorithms were first tested in simulation which required models of each type of sensor to be developed with the inclusion of measurement noise. Both algorithms proved to be fairly accurate over short time periods, with the OFS-based algorithm experiencing only small amounts of drift in the positional estimates over extended time periods. However a more significant limitation of the OFS-based algorithm was the increased variability of velocity estimates when operating at larger distances above the surface - resulting in more significant positional drift. In summary, the GPS-based algorithm resulted in more accurate positional estimates, but the OFS-based algorithm was adequately accurate with little drift when positioned at a hover close to the surface.\\

Finally, a hexacopter platform was equipped with sensors and wireless capability to enable autonomous flights. The software onboard the hexacopter was ArduPilot - a versatile open source flight control architecture. Using this software, the aircraft performed multiple flight tests and the data gathered by the onboard sensors was stored on an SD card for post-flight processing. This data was then processed using the state estimation algorithms in order to validate their effectiveness. However, the usefulness of the data was limited by the data logging rates being significantly slower than the sensor sampling rates. The state estimation techniques showed promise but would require further testing on hardware to ensure their accuracy.


Overall, the project was successful in deriving a mathematical model of the aircraft, validating a robust control system against that model and demonstrating accurate extended Kalman filter-based state estimation in simulation. A hardware platform was also configured with software as well as sensors, additional processing and wireless capability, resulting in successful autonomous flights.


\section{Future Extensions}
Automation of aircraft is a rapidly expanding area of research and there are many possible extensions to this project. A number of examples possible additions to this project are listed:

\begin{itemize}
\item Implement the backstepping control system on the hardware platform.
\item Account for motor/propeller failure events during flight.
\item Extend sensor fusion algorithm to handle sensor failure.
\item Implement obstacle avoidance.
\item Analyse the usefulness of additional sensors e.g differential pressure sensor (for airspeed).
\item Consider the impacts of other effects e.g air resistance and gyroscopic effects.
\item Consider slew rates of actuators, propeller inertia, etc.
\item Develop a quaternion-based model to avoid singularities.
\item Account for sensor biases in state estimation algorithms.
\item Develop a path generation block as part of the control system.


\end{itemize}

\clearpage


