
\chapter{Introduction}
\textit{This chapter introduces the topic of this thesis - unmanned aerial vehicles - and discusses their contemporary applications. In doing so, this chapter establishes the motivation for this thesis and potential applications of the results. An outline of the content in each chapter is also presented.}\\

\section{Project Motivation}
Unmanned aerial vehicles (UAVs) allow tasks to be undertaken in difficult to access areas or risky environments without endangering human life. In many current applications, a human operator is required for remote control of the vehicle. However, with current technological advances in automation, it is becoming viable to use UAVs which operate completely autonomously i.e. without a human operator. This has the potential to improve the safety, efficiency, reliability, and cost of many processes. However, autonomous control of an aerial vehicle is a complex problem requiring the consideration of many areas of research, including but not limited to: modelling, control law design, sensor fusion, navigation, collision avoidance and fault tolerance.\\

UAVs were originally developed and used mainly by military organisations, beginning with the Sperry Aerial Torpedo developed during World War I for the US Navy, a radio-controlled biplane designed to be used as a flying bomb \cite{Stoff2001}. This was followed by decades of development leading to vehicles used for aerial target practice, reconnaissance, and remote bombing. However, in recent decades there has been an increased interest in the use of UAVs for both commercial and scientific applications. Commercial applications include surveying, aerial photography/videography, package delivery and pesticide spraying. UAVs are increasingly being utilised in many industries from mining to agriculture. Technological advances in computing and electronics have decreased the cost of UAVs and this has resulted in a large selection of vehicles being more widely available to hobbyists, not just large companies. Scientific research has also benefited from the use of UAVs, with applications including monitoring bodies of water for algal blooms, identifying plant species, monitoring coastal erosion and tracking/counting animal populations. One recent study \cite{Raoult2018} uses UAVs to track aquatic vertebrates without using invasive tagging methods. Even more recently, the first unmanned flights on another planet were completed by NASA’s Ingenuity helicopter on Mars \cite{Johnson2021}, demonstrating that UAV technology could be a viable option for extra-terrestrial exploration in the near future. Further, NASA has a mission planned for launch in 2026 to send a multi-rotor aircraft, named Dragonfly, to explore Titan – the largest moon of Saturn \cite{Hautaluoma2019}. In such applications communication with the vehicles is extremely limited due to the astronomical scale of the distance between the ground station and the vehicle, as well as the potential for obstructions and interference. Therefore the vehicle must be able to operate autonomously without human intervention. Also, since the conditions on other planets are not as well characterised as those on Earth, the control system must be robust to model uncertainties and disturbances. Another key difference is the availability of positioning data - i.e. there are no GPS satellite constellations orbiting Mars or Titan, so the UAV must have another method of position estimation. Likewise, many environments on Earth have limited or unreliable GPS data e.g. indoors, underground or hostile territory with GPS jamming devices. The versatility of UAVs results in an almost limitless number of applications for their use.\\


UAVs come in many different configurations with the main classifications being fixed-wing and rotary-wing, however there also exist some hybrid configurations. Rotorcraft have the advantage of being able to perform Vertical Take-Off and Landing (VTOL), i.e. they do not require a runway for take off and landing. Also, rotorcraft are able to hover in a fixed position, as opposed to fixed wing vehicles which need to continually move to generate lift. On the other hand, fixed-wing vehicles are generally able to travel at much higher speeds and can fly at higher altitudes. Hybrid configurations generally combine the benefits of the two types - resulting in a vehicle which is capable of VTOL and hovering, but also able to travel at high speeds and altitudes when necessary. A well-known example of a vehicle with hybrid capabilities is the V-22 Osprey. This thesis will explore rotorcraft configurations due to their versatility and widespread commercial availability.\\

The defining characteristic of rotorcraft configurations is the number of rotors present on the vehicle. There is an almost limitless number of configurations from single rotor vehicles (helicopters) up to octocopters (8 rotors) and beyond. The topic of this thesis is centred around unmanned multi-rotor vehicles and will begin with an overview of the basic dynamics of these systems. These ideas will then be extended to the specific configuration of a six-rotor vehicle - the hexacopter. Once an accurate model is developed, it will be used for simulation to enable the development and testing of a robust control system. A method for state estimation will then be explored with a focus on position estimation in GPS denied environments.



\section{Thesis Outline}
This thesis is divided into seven chapters. The contents of the chapters (including this one) are summarised below:
\begin{itemize}
\item Chapter One introduces the main concepts surrounding unmanned aerial vehicles and outlines the motivation for the project.
\item Chapter Two investigates the technical background necessary for the concepts covered in this thesis. The key theory is summarised and useful formulae are presented and explained. The current literature around modelling, control and state estimation for multi-rotor vehicles is also evaluated.
\item Chapter Three develops a comprehensive mathematical model to describe the dynamics of a general multi-rotor vehicle. This is then extended to a specific configuration  with six propellers known as a hexacopter. This model is simulated using MATLAB Simulink to enable the development and verification of control systems in the following chapter.
\item Chapter Four explores the control of the hexacopter through the development and testing of a number of progressively more complex control systems. The chapter begins with the development of a PID controller. Next, backstepping control theory is used to stabilise the system. Then integral action is added to the backstepping controller and finally the issue presented by actuator saturation is addressed.
\item Chapter Five addresses the estimation of the systems states using extended Kalman filtering techniques. First, an algorithm which relies upon GPS data is developed. Next, an alternative algorithm which uses an optical flow sensor is developed and tested for use in situations where GPS signal is not available. The two algorithms are then compared in simulation.
\item Chapter Six discusses a custom hexacopter hardware platform. Additional sensors were connected to the flight controller and a companion computer was added to enable wireless communication. Software was configured onboard the flight controller, the companion computer and on a ground system computer to allow autonomous flights to be planned and carried out. This hexacopter is then used for gathering flight data for processing with the state estimator algorithms.
\item Chapter Seven, the final chapter, summarises the results of the project and discusses future extensions.
\end{itemize}

\clearpage


