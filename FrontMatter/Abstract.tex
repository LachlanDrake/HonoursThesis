
\chapter*{Abstract}\label{Abstract}
\addcontentsline{toc}{chapter}{Abstract}
This thesis outlines modelling, control and state estimation techniques for autonomous flight of multi-rotor unmanned aerial vehicles (UAV). First, an accurate mathematical model of the system was established and model parameters were measured or estimated. This model was implemented in MATLAB Simulink, such that various control techniques could be simulated. A stable and accurate control system was progressively developed by iterative testing and improvement. Initially PID control was implemented, followed by a backstepping controller and finally an integral backstepping controller. Actuator saturation was then accounted for with the use of a sigmoid function. This control system was tested and found to adequately stabilise the system and also allow positional commands to be followed. Next, state estimation techniques were explored with the use of various sensors in an extended Kalman filter (EKF) algorithm. In particular two algorithms were developed and tested, with one using Global Positioning System (GPS) technology and the other using an optical flow sensor (OFS) for use in GPS-limited environments. Finally, a UAV hardware platform was configured with open source flight control software (ArduPilot) and flown such that the EKF algorithm could be tested on real flight data gathered from the sensors. The simulation results demonstrated that the OFS-based algorithm was effective in estimating position with minimal drift when the vehicle is in a hover close to the ground. However, the GPS-based algorithm produced significantly more accurate position results, both in simulation and in hardware. 
\clearpage


